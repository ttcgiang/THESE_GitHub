%% LyX 2.1.3 created this file.  For more info, see http://www.lyx.org/.
%% Do not edit unless you really know what you are doing.
\documentclass[english]{amsart}
\usepackage[T1]{fontenc}
\usepackage[utf8x]{inputenc}
\setcounter{secnumdepth}{3}
\setcounter{tocdepth}{3}
\usepackage{amsthm}
\usepackage{amsmath}
\usepackage{amssymb}

\makeatletter
%%%%%%%%%%%%%%%%%%%%%%%%%%%%%% Textclass specific LaTeX commands.
\numberwithin{equation}{section}
\numberwithin{figure}{section}
\theoremstyle{plain}
\newtheorem{thm}{\protect\theoremname}
  \theoremstyle{definition}
  \newtheorem{defn}[thm]{\protect\definitionname}
  \theoremstyle{plain}
  \newtheorem{prop}[thm]{\protect\propositionname}

%%%%%%%%%%%%%%%%%%%%%%%%%%%%%% User specified LaTeX commands.
\usepackage{babel}

\makeatother

\usepackage{babel}
  \providecommand{\definitionname}{Definition}
  \providecommand{\propositionname}{Proposition}
\providecommand{\theoremname}{Theorem}

\begin{document}

\title{Probabilistic derivation of multi-population epidemic model\\
 with $\beta_{ijk}=-\kappa_{j}\log(1-c_{ik})$}


\author{Giang, Yann, Marc et JD}
\maketitle
\begin{defn}
Durant le petit intervalle de temps $\delta t$, chaque individu natif
de la ville $i$ visite \textbf{une seule} ville $j$ (avec probabilité
$\rho_{ij}$) et rencontrera \textbf{en moyenne }$\kappa_{j}$ individus.
Ces individus proviennent de toutes les villes.
\end{defn}

\section{Notation}


\subsubsection*{Notation :}

Here, we present list of sets and events describing the state of the
system at time $t$ : 
\begin{itemize}
\item $C_{i}$ is the set of all individuals born in subpopulation $i$. 
\item $V_{i,t}$ is the set of all individuals physically located in subpopulation
$i$ from time $t$ to time $t+\delta t$. This includes foreigners
traveling in subpopulation $i$ at time $t$, and all natives from
subpopulation $i$ which are not traveling abroad at time $t$. 
\item $S_{t},E_{t},I_{t},R_{t}$ are the sets of all individuals respectively
susceptible, exposed, infected and recovered at time $t$. Note that
these set include individuals from all subpopulations. 
\item $S_{i,t},E_{i,t},I_{i,t},R_{i,t}$ are the same sets, restricted to
natives of subpopulation $i$. So formally, $S_{i,t}=S_{t}\cap C_{i}$,
$E_{i,t}=E_{t}\cap C_{i}$, $I_{i,t}=I_{t}\cap C_{i}$, and $R_{i,t}=R_{t}\cap C_{i}$. 
\item $Transmit(y,x)$ is an event indicating that individual $x$ gets
infected by individual $y$ which was already infected 
\item $c_{i,k}$ is the probability that a susceptible individual native
from $i$ being in contact with another infected individual native
from $k$ gets infected. 
\item $\kappa_{j}$ is the average number of contacts per unit of time a
susceptible will have when visiting city $j$. 
\item $\xi_{jk}$ refers to the probability that an individual $y$ meeting
$x$ in $C_{j}$ comes from $C_{k}$.
\item $\rho_{i,j}$, the probability that an individual from subpopulation
$i$ visits subpopulation $j$. Of course, $\sum_{j=1}^{M}\rho_{ij}=1$.\end{itemize}
\begin{prop}
The coefficient $\kappa$ should also depend on $i$, because an individual
native from city $i$ meets more people in his own city than abroad
($\kappa_{i,i}>\kappa_{i,j}$). 
\end{prop}

\section{The basics (nearly same as in earlier versions)}

\textbf{Let us write a probabilistic formulation of $\frac{dE_{i}}{dt}$
:}

One general question is always posed ``how does the population of
exposed individuals of subpopulation $i$ evolve ?''. For the sake
of simplicity, in the process of transmission of the SEIR model, we
focus on the incidence and we assume for now that the latent period
and the recovery rate, repectively $\mu=\sigma=0$. Thus, we write
a probabilistic formulation of $\frac{dE_{i}}{dt}$. Assuming the
time is discrete, we have $\frac{dE_{i}}{dt}\approx\mathbb{E}\left[E_{i,t+1}\setminus E_{i,t}\right]$.
Then,

\begin{eqnarray*}
\mathbb{E}\left[E_{i,t+1}\setminus E_{i,t}\right] & = & \mathbb{E}\left[E_{i,t+1}\cap S_{i,t}\right]\\
 & = & \sum_{x\in C_{i}}Pr\left[x\in E_{t+1}\wedge x\in S_{t}\right]\\
 & = & \sum_{x\in C_{i}}Pr\left[x\in S_{t}\right]*Pr\left[x\in E_{t+1}\mid x\in S_{t}\right]\\
 & = & Pr_{x\sim\mathcal{X}_{i}}\left[x\in E_{t+1}\mid x\in S_{t}\right]*\sum_{x\in C_{i}}Pr\left[x\in S_{t}\right]\\
 & = & |S_{i,t}|\times Pr_{x\sim\mathcal{X}_{i}}\left[x\in E_{t+1}\mid x\in S_{t}\right]
\end{eqnarray*}


Assume there are $M$ cities. An individual $x$ of the subpopulation
$i$ may be visiting another subpopulation, or staying in its own
subpopulation. Applying the law of total probabilities, we get:

\begin{eqnarray*}
Pr_{x\sim\mathcal{X}_{i}}\left[x\in E_{t+dt}\mid x\in S_{t}\right] & = & \sum_{j=1}^{M}Pr_{x\sim\mathcal{X}_{i}}\left[x\in E_{t+dt}\wedge x\in V_{j,t}\mid x\in S_{t}\right]\\
 & = & \sum_{j=1}^{M}Pr_{x\sim\mathcal{X}_{i}}\left[x\in E_{t+dt}\mid x\in S_{t}\wedge x\in V_{j,t}\right].Pr_{x\sim\mathcal{X}_{i}}\left[x\in V_{j,t}\right]\\
 &  & \sum_{j=1}^{M}Pr_{x\sim\mathcal{X}_{i}}\left[x\in E_{t+dt}\mid x\in S_{t}\wedge x\in V_{j,t}\right]\times\rho_{ij}
\end{eqnarray*}


Where $\rho_{i,j}=Pr_{x\sim\mathcal{X}_{i}}\left[x\in V_{j,t}\right]$,
the probability that an individual from subpopulation $i$ visits
subpopulation $j$. Of course, $\sum_{j=1}^{M}\rho_{ij}=1$.


\section{Study of case where agent $x$ native from city $i$ visits city
$j$}

Here, we look at the probability that a susceptible $x\sim\mathcal{X}_{i}$
visiting $j$ gets infected or not after $\delta t$ time steps. Let
$\mathcal{Y}$ be the uniform distribution over $V_{j,t}$. (\textbf{IMPORTANT})
The correct mathematical approach for this would be to assume that
for each city $k$, the number of people native from $k$ that we
meet during $\delta t$ follows a Poisson process. So both the number
of people we meet and the number of infected people we meet during
$\delta t$ should be random variables.

In the approach described in \cite{keeling2011}, the authors did
not do this. They assumed that both the number of people we meet and
the number of infected people we meet \emph{are fixed} (otherwise
the maths they write would have been different). We will call this
the ``Keeling \& Rohani'' interpretation

We introduce an alternative approximation, where we assume that the
number $\kappa$ of people we meet during $\delta t$ is \emph{fixed},
but each of these people has \emph{some probability} to be infected.
This is an \emph{in-between interpretation}, easier than the Poisson
process maths, but better than Keeling\&Rohani's one. We will call
this the ``Yann-Giang'' interpretation.


\subsection{The ``Yann-Giang'' interpretation}
\begin{prop}
Agent $x$ meets \emph{exactly} $\kappa_{j}$ other individuals, and
each of these individuals has a probability $\frac{\left|I_{k,t}\right|}{N_{k}}$
of being infected, where $k$ is its native city. Let $y_{1}\ldots y_{\kappa_{j}}$
be the individuals that $x$ meets. We get:
\end{prop}
\begin{eqnarray*}
 &  & Pr_{x\sim\mathcal{X}_{i}}\left[x\in S_{t+\delta t}\mid x\in S_{t}\wedge x\in V_{j,t}\right]\\
 & = & Pr_{x\sim\mathcal{X}_{i},y_{1}\ldots,y_{\kappa_{j}}\sim\mathcal{Y}}\left[\bigwedge_{p=1}^{\kappa_{j}}\neg\left(y_{p}\in I_{t}\wedge Transmit(y_{p},x)\right)\mid x\in S_{t}\wedge x\in V_{j,t}\right]
\end{eqnarray*}


So we have:

\begin{eqnarray*}
 &  & Pr_{x\sim\mathcal{X}_{i}}\left[x\in S_{t+\delta t}\mid x\in S_{t}\wedge x\in V_{j,t}\right]\\
 & = & Pr_{x\sim\mathcal{X}_{i},y\sim\mathcal{Y}}\left[\neg\left(y\in I_{t}\wedge Transmit(y,x)\right)\mid x\in S_{t}\wedge x\in V_{j,t}\right]^{\kappa_{j}\delta t}
\end{eqnarray*}


As in the earlier versions of the document, we have:
\begin{itemize}
\item The probability so that a susceptible individual $x$ is infected
by an infected individual $y$ :
\end{itemize}
\begin{eqnarray*}
 &  & Pr_{x\sim\mathcal{X}_{i},y\sim\mathcal{Y}}\left[y\in I_{t}\wedge Transmit(y,x)\mid x\in S_{t}\wedge x\in V_{j,t}\right]\\
 & = & \sum_{k=1}^{M}Pr_{x\sim\mathcal{X}_{i},y\sim\mathcal{Y}}\left[y\in I_{t}\wedge Transmit(y,x)\mid x\in S_{t}\wedge x\in V_{j,t}\wedge y\in C_{k}\right].Pr_{y\sim\mathcal{Y}}\left(y\in C_{k}\right)\\
 & = & \sum_{k=1}^{M}\left\{ Pr_{x\sim\mathcal{X}_{i},y\sim\mathcal{X}_{k}}\left[y\in I_{t}\mid x\in S_{t}\wedge x\in V_{j,t}\right]\right.\\
 &  & \,\,\,\,\,\left.\times Pr_{x\sim\mathcal{X}_{i},y\sim\mathcal{X}_{k}}\left[Transmit(y,x)\mid y\in I_{t}\wedge x\in S_{t}\wedge x\in V_{j,t}\wedge y\in C_{k}\right]\times Pr_{y\sim\mathcal{Y}}\left(y\in C_{k}\right)\right\} \\
 & = & \sum_{k=1}^{M}\left(\frac{\left|I_{k,t}\right|}{N_{k}}\times c_{ik}\times\xi_{jk}\right)
\end{eqnarray*}


$\xi_{jk}=\frac{N_{k}\rho_{kj}}{\sum_{v=1}^{M}N_{v}\rho_{vj}}$ refers
to the probability that an individual $y$ meeting $x$ in $C_{j}$
comes from $C_{k}$.
\begin{itemize}
\item Therefore, the probability so that a susceptible individual $x$ is
not infected by an infected individual $y$ :
\end{itemize}
\[
1-\sum_{k=1}^{M}\left(\frac{\left|I_{k,t}\right|}{N_{k}}\times c_{ik}\times\xi_{jk}\right)
\]

\begin{itemize}
\item The probability so that a susceptible individual $x$ is not infected
after $\kappa_{j}$ contacts per unit time $\delta t$.
\end{itemize}
\[
\left[1-\sum_{k=1}^{M}\left(\frac{\left|I_{k,t}\right|}{N_{k}}\times c_{ik}\times\xi_{jk}\right)\right]^{\kappa_{j}\delta t}
\]

\begin{itemize}
\item Thus, the probability so that a susceptible individual $x$ becomes
infected after $\kappa_{j}$ contacts per unit time $\delta t$.
\end{itemize}
\begin{eqnarray*}
Pr_{x\sim\mathcal{X}_{i}}\left[x\in E_{t+\delta t}\mid x\in S_{t}\wedge x\in V_{j,t}\right] & = & \left[1-\sum_{k=1}^{M}\left(\frac{\left|I_{k,t}\right|}{N_{k}}\times c_{ik}\times\xi_{jk}\right)\right]^{\kappa_{j}\delta t}
\end{eqnarray*}


We now apply the \emph{log} approximation which consists in approximating
$1-(1-u)^{v}$ by $v\log(1-u)$:

\begin{eqnarray*}
Pr_{x\sim\mathcal{X}_{i}}\left[x\in E_{t+\delta t}\mid x\in S_{t}\wedge x\in V_{j,t}\right] & = & -\kappa_{j}\delta t\log\left[1-\sum_{k=1}^{M}\left(\frac{\left|I_{k,t}\right|}{N_{k}}\times c_{ik}\times\xi_{jk}\right)\right]
\end{eqnarray*}


So...

\[
\frac{dPr_{x\sim\mathcal{X}_{i}}\left[x\in E_{t+dt}\mid x\in S_{t}\wedge x\in V_{j,t}\right]}{dt}\simeq-\kappa_{j}\log\left[1-\sum_{k=1}^{M}\left(\frac{\left|I_{k,t}\right|}{N_{k}}\times c_{ik}\times\xi_{jk}\right)\right]
\]


Overall, we have the formule of the infection force as follows:

\[
\lambda_{i}=\sum_{j}\rho_{ij}\kappa_{j}\log\left[1-\sum_{k=1}^{M}\left(\frac{\left|I_{k,t}\right|}{N_{k}}\times c_{ik}\times\xi_{jk}\right)\right]
\]


Thus, in the forcing case, the contact rate is in the sinusoidal form
as follows :

\[
\kappa_{i}=\kappa_{i0}(1+\kappa_{i1}cos(\frac{2\pi t}{T}+\phi_{i})
\]



\subsection{``Keeling \& Rohani'' Interpretation}
\begin{prop}
Agent $x$ meets \emph{exactly} $\kappa_{j}\delta t\xi_{jk}\frac{\left|I_{k,t}\right|}{N_{k}}$
other infected individuals native from city $k$. (UGLY!!!)
\end{prop}
Let $l_{k}=\kappa_{j}\delta t\xi_{jk}\frac{\left|I_{k,t}\right|}{N_{k}}$.
Let $y_{1}^{k}\ldots y_{l_{k}}^{k}$ be the infected individuals native
from $k$ that our individual $x$ meets between $t$ and $t+\delta t$.
\begin{eqnarray*}
 &  & Pr_{x\sim\mathcal{X}_{i}}\left[x\in S_{t+\delta t}\mid x\in S_{t}\wedge x\in V_{j,t}\right]\\
 & = & Pr_{x\sim\mathcal{X}_{i}}\left[\bigwedge_{\begin{array}{c}
k=1\ldots M\\
p=1\ldots l_{k}
\end{array}}\neg\left(Transmit(y_{p}^{k},x)\right)\mid x\in S_{t}\wedge x\in V_{j,t}\right]\\
 & = & \prod_{k=1}^{M}Pr_{x\sim\mathcal{X}_{i}}\left[\bigwedge_{p=1\ldots l_{k}}\neg\left(Transmit(y_{p}^{k},x)\right)\mid x\in S_{t}\wedge x\in V_{j,t}\right]\\
 & = & \prod_{k=1}^{M}\left(1-c_{ik}\right)^{\kappa_{j}\delta t\xi_{jk}\frac{\left|I_{k,t}\right|}{N_{k}}}
\end{eqnarray*}


We plug this back into the previous formula, and we get:

\begin{eqnarray*}
Pr_{x\sim\mathcal{X}_{i}}\left[x\in E_{t+\delta t}\mid x\in S_{t}\wedge x\in V_{j,t}\right] & = & 1-\prod_{k=1}^{M}\left(1-c_{ik}\right)^{\kappa_{j}\xi_{jk}\frac{\left|I_{k,t}\right|}{N_{k}}\delta t}
\end{eqnarray*}


The first order approximation of $1-\prod_{k=1}^{M}(1-c_{ik})^{v_{k}}$
is $\sum_{k=1}^{M}-v_{k}\log(1-c_{ik})$. Applying this approximation
here, we get:

\begin{eqnarray*}
Pr_{x\sim\mathcal{X}_{i}}\left[x\in E_{t+\delta t}\mid x\in S_{t}\wedge x\in V_{j,t}\right] & \simeq & \delta t\sum_{k=1}^{M}\left(-\kappa_{j}\xi_{jk}\frac{\left|I_{k,t}\right|}{N_{k}}\log\left(1-c_{ik}\right)\right)
\end{eqnarray*}


Define $\beta_{ijk}=-\kappa_{j}\log\left(1-c_{ik}\right)$, let $\delta t$
converge to zero, and we get:

\[
\frac{dPr_{x\sim\mathcal{X}_{i}}\left[x\in E_{t+dt}\mid x\in S_{t}\wedge x\in V_{j,t}\right]}{dt}\simeq\sum_{k=1}^{M}\left(\xi_{jk}\frac{\left|I_{k,t}\right|}{N_{k}}\beta_{ijk}\right)
\]


If there is only one city $i$, then we fall back to the formula of
\cite{keeling2011} 

We have : 

\[
\beta_{i}=-\kappa_{i}\log\left(1-c_{i}\right)
\]


\[
\frac{d}{dt}\mathbb{E}\left[\left|E_{i,t+dt}-E_{i,t}\right|\right]\simeq-\left|S_{i,t}\right|\left(\frac{\left|I_{i}\right|}{N_{i}}\beta_{i}\right)
\]



\section{Final Formula}

We simply have to plug in the probability $\rho_{ij}$ that $i$ visits
$j$.

We get, for the ``Yann-Giang'' interpretation :

\[
\frac{d}{dt}\mathbb{E}\left[\left|E_{i,t+dt}-E_{i,t}\right|\right]\simeq-\left|S_{i,t}\right|\sum_{j}\rho_{ij}\kappa_{j}\log\left[1-\sum_{k=1}^{M}\left(\frac{\left|I_{k,t}\right|}{N_{k}}\times c_{ik}\times\xi_{jk}\right)\right]
\]


And for the ``Keeling \& Rohani'' Interpretation :

\[
\frac{d}{dt}\mathbb{E}\left[\left|E_{i,t+dt}-E_{i,t}\right|\right]\simeq-\left|S_{i,t}\right|\sum_{j}\rho_{ij}\sum_{k=1}^{M}\left(\xi_{jk}\frac{\left|I_{k,t}\right|}{N_{k}}\beta_{ijk}\right)
\]

\begin{thebibliography}{keeling2011}
\bibitem[keeling2011]{keeling2011}Matt J. Keeling,Pejman Rohani (2011)
Modeling Infectious Diseases in Humans and Animals\end{thebibliography}

\end{document}
